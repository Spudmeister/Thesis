\chapter{Limit Setting}
Cross-section exclusion limits are set at the 95\% confidence level for this analysis. The calculation of these levels and how the results are interpreted are discussed in this chapter. Cross-section limits are calculated use the \CLs technique, which is a modified frequentist approach and the standard technique for limits set by experiments at the \LHC \cite{CLS2002},\cite{smallStatCLs}. Limits are presented one dimensionally as a function of \WR mass for both the boosted and resolved \NR case. Two dimensional limits are also shown with resolved and boosted treated separately and combined.

This analysis considers each bin in the final \WR mass distribution simultaneously. The binning choices for the resolved and boosted analysis are discussed in chapter \ref{ch:bg_estim}. Inter-bin correlations improve the ability to distinguish signal from background given the \WR mass's peaking behavior. For the heaviest \WR mass that we can set a limit for, however, the last bin influences the limit far more than any other. 
\section{The \CLs Technique}

\subsection{Overview}
The \CLs technique is a modified frequentist method for setting exclusion limits. The frequentist limits used here answer the question, ``How probable is this observation based on a given model''. This is in contrast to Bayesian statistics, which calculate a probability of a model given certain results.

Fundamentally, setting exclusion limits is a form of hypothesis testing. We can consider two hypotheses. One is that the observed data results from standard model background events. This will be called the null hypothesis, $H_{0}$. The second hypothesis is that the data results from a combination of standard model background events and a new signal, $H_{\mu}$. The variable $\mu$ will be used to represent the amount of signal strength. For this analysis, the signal would be a \WR boson. Now some variable must be selected to allow us to discern between the two hypotheses. This variable is called the test statistic. The total number of events measured in data after some event requirements is an example of a test statistic and is a simplified version of what this analysis uses.

The probability distribution of the test statistic, defined as $q\left(X\right)$, has to be estimated for each of the hypotheses, background only ($H_{0}$) and background + signal ($H_{\mu}$). To estimate the test statistic's distribution for the two hypotheses, toy MC can be used to create many different possible outcomes for $q\left(X\right)$ including all of the uncertainties in the analysis. This analysis uses an approximation of the toy MC technique instead, as the full toy MC technique takes a significant amount of computation time. The difference between these two techniques was studied and is shown in section \ref{sec:asymp_limits}. Once the probability distribution is determined, the probability that the result is caused by background only, $H_{0}$, is
\begin{equation}
    \mathrm{CL}_{b}
    \equiv
    \int\limits_{q\left(X\right)}^{\infty}f\left(0\right)\mathrm{dq}.
\end{equation}

The confidence level for the background only hypothesis, $H_{0}$, is $\mathrm{CL}_{b}$. The probability that the null hypothesis explains a disagreement at least as large as the disagreement between the measured data and the expected background only result is defined as $1-\mathrm{CL}_{b}$. This value is also called the p-value. The confidence level for the signal + background hypothesis, $\mathrm{CL}_{s+b}$, can be calculated as
\begin{equation}
    \mathrm{CL}_{s+b}
    \equiv
    \int\limits_{q\left(X\right)}^{\infty}f\left(\mu\right)\mathrm{dq}.
\end{equation}

The values of $\mathrm{CL}_{s+b}$ and $\mathrm{CL}_{s+b}$ can be used to discern between the null hypothesis, $H_{0}$, and the signal + background hypothesis, $H_{\mu}$. The threshold for rejecting the null hypothesis, or the alternative hypothesis were set by \CMS before any analyses commenced. This removes a potential source of bias in the results. The famous ``$5\sigma$'' threshold is defined as $1-\mathrm{CL}_{b} < 2.87\times10^{-7}$.

Often in searches for new phenomena, no clear evidence for discovery exists in data. Exclusion limits are calculated instead. These limits are calculated based on the confidence interval calculations defined above and are commonly set at the 95th percentile. An exclusion limit is designed to exclude signal by requiring that the probability that the observed data can be described by a background only hypothesis is less than 5\%. This works out to be $\mathrm{CL}_{s+b}<1-0.95$. Calculating limits in this fashion gives problematic results when backgrounds are much larger than expected signal. The \CLs technique handles this by normalizing the signal + background confidence level with the background only confidence level.
\begin{equation}
    \mathrm{CL}_{S}
    \equiv
    \frac{\mathrm{CL}_{s+b}}{\mathrm{CL}_{b}} < 1-0.95.
\end{equation}

This gives the modified frequentist confidence limit, \CLs. \CLs isn't a true confidence level as it is designed to give values relative to the background confidence level, which are by construction more conservative than $\mathrm{CL}_{s+b}$ alone.

\subsection{Higgs Combine}

\subsection{Asymptotic Limits}
\label{sec:asymp_limits}
A comparison of the asymptotic limits with toy MC limits

\subsection{One Dimensional Limits}
To construct one dimensional limits, the relationship of the \NR to the \WR mass is fixed. For the resolved limit, only the \NR with a mass of half of the \WR mass are considered. For the boosted limit, the \NR mass is fixed at \ensuremath{\SI{100}{\GeV}}. This is the lowest mass \NR considered at every \WR mass point, and so it represents the most boosted \WR case for each \WR mass.
\subsubsection{Resolved}

\subsubsection{Boosted}

\subsection{Two Dimensional Limits}
\subsubsection{Resolved}
\subsubsection{Boosted}
\subsubsection{Combination}



\label{ch:limit_setting}